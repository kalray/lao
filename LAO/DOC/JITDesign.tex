\documentclass[11pt,a4paper]{article}
\usepackage{palatino,epsfig}
\usepackage[latin1]{inputenc}
\usepackage[T1]{fontenc}
\usepackage{latexsym}
\usepackage{pstcol}

\title{Just-In-Time Compiler Design
for a .NET Embedded Translation System}
\author{Beno\^{\i}t Dupont de Dinechin}

\begin{document}
\maketitle

\begin{abstract}

We discuss the design of a Just-In-Time (JIT) for an embedded program
translation system. This system uses Microsoft .NET CIL byte codes as input and
produces binary executable code for embedded processors such as the ST200 and
ARM families.

\end{abstract}


\section{Design Objectives and Constraints}

The JIT design objectives and constraints are set by the \textbf{cli-tech}
project of AST computing in STMicroelectronics: \begin{description}

\item[R1] The proposed JIT compiler is running CIL code that has been produced
from: Microsoft Visual Studio when compiling standard C and C++ code under user
option \verb|/clr:pure|; the LCC compiler with its MSIL code generator; an
upcoming CIL generator developed by AST Computing on top of the GCC4 compiler.

\item[R2] The JIT compiler will be used in install-time, load-time, and later
run-time, configurations on an embedded device that comprises a micro-controller
and one or more media processors. The JIT compiler will target the media
processor(s) but may execute on the micro-controller.

\item[R3] The JIT compiler will focus on high performance of generated code,
relying on the upstream compiler to specifically optimize and annotate the CIL
code it produces. Because the applications considered are media processing
algorithms written in C or C++, the JIT compiler does not have to deal with
managed data (e.g. support garbage collection by an underlying virtual machine).

\item[R4] The JIT compiler source code should be mostly target-independent. Most
of the target system dependencies should be captured in the tables that are
produced by the Machine Description System (MDS).

\item[R5] The JIT compiler should be able to recover from low memory errors.
Whenever the memory required for a compiler optimization cannot be obtained, the
JIT compiler should signal the problem to the environment, then try again the
optimization dumbed down.

\item[R6] For the JIT development considered here, the speed of code generation
should be in the range of 10,000 to 100,000 processor cycles per CIL byte code.

\end{description}

There is no need for this JIT to support the highest possible code generation
speed at expense of code quality. If required, such high-speed JIT compilation
will be performed by a separate program called JIT0. From published numbers of
JIT research and production projects, it appears that the highest possible code
generation speed in a JIT0 is about 1000 processor cycles average per CIL byte
code.


\section{Proposed JIT Software Architecture}


\subsection{Region-Based Compilation}

Control of the resources (time and space) needed by performance-oriented
compilation can be achieved by scoping the optimizations to \emph{code regions},
instead of applying them to a whole procedure at a time
\cite{Hank:1995:MICRO,Way:2000:PACT}. Recent work by the IBM JVM team
\cite{Suganuma:2003:PLDI} extends the applicability of region-based compilation
to dynamic optimizations in a Java JIT by speculatively ignoring the data-flow
effects of seldom executed code regions. In this setting, they demonstrate that
region-based compilation not only reduces the compilation resource requirements,
but also provides significant performance improvements of the resulting code.

For the JIT we propose, code regions implemented as the CodeRegion class provide
the scope for most analyzes and transformation, in particular SSA form building
and register allocation. These CodeRegion(s) are a partition of a Procedure
control-flow graph, possibly after some inlining has been performed.  To be more
specific, the CodeRegion(s) we propose are not hierarchical. However,
CodeRegion(s) are containers for optimization-dependent partitions of the
control-flow graph, in particular IF-conversion regions, loop optimization
regions, extended basic blocks and instruction scheduling regions.

In order to support exceptions in C++, the JIT has to represent the nesting of
\emph{try/catch/finally} program scopes. This nesting which is hierarchical must
be represented by a structure distinct from CodeRegion(s). The constraints on
\emph{try/catch/finally} program scopes on CodeRegion boundaries is yet to be
specified.


\subsection{SSA Form Program Representations}

A recent trend in compiler code generator design is the use of SSA form on
machine operators \cite{Leung:1999:PLDI,Dinechin:2000:CASES} in static
compilers. Another is the use of SSA form with generic code followed by a code
selection to a machine-level program representation, as exemplified in the Intel
StarJIT compiler \cite{Tabatabai:2003:INTELTJ}. A more pervasive use of SSA form
in JIT compilation is exemplified by the Java HotSpot Server Compiler
\cite{Paleczny:2001:JVM}. The usual perception with the SSA form is that it
expensive to construct and destruct, even though it makes optimizations less
costly and more effective once available. It is interesting to note that, for
the IBM JVM team, even the SSA form optimizations themselves are too expensive
for an optimizing JIT \cite{Suganuma:2000:IBMSJ}.

For the JIT we propose, we use the SSA form pervasively, both at the generic
code level and the machine code level. The SSA form at the generic code level
means a SSA form whose node operators are target independent. This SSA form is
similar to what exists in the target-independent optimization stages of static
optimizing compilers, such as \textbf{wopt} in the Open64 compiler. However, the
generic operators we propose closely match a subset of the CIL opcodes in order
to simplify the translation from CIL object format. At the code selection stage,
the SSA form remains except that the generic operators patterns are matched an
expanded to machine operators.

In order to save the compilation resources while enabling high code quality, we
adopt the following key design principles: \begin{itemize}

\item SSA form optimizations and analyzes should be done whenever possible on
the generic SSA form.

This includes at least: method inlining and loop unrolling; cleanup with
constant propagation and dead code elimination; bit-width analysis to prepare
the code selection. Some cleanup of the machine SSA form is also required.

\item Eliminate the need for explicit interference graph construction in
SSA form destruction, register coalescing and allocation.

The interference graph is normally required for these steps. However, in case
\emph{strict SSA form} is maintained\footnote{Definitions by Phi-functions
dominate their uses. This is the case after SSA form construction but may not be
after PRE or even value numbering.} interferences can be quickly checked by
combining dominance and liveness information \cite{Budimlic:2002:PLDI}. This
interference test has been combined with the SSA form destruction and variable
coalescing of Sreedhar et al. \cite{Sreedhar:1999:SAS} in the LAO. Last, using
the most advanced linear-scan allocator as developed for the Java HotSpot client
compiler \cite{Wimmer:2005:VEE}, register allocation can be achieved in almost
linear time without relying on an explicit interference graph.

Also, a research objective for the proposed JIT is to achieve register
allocation for machine code SSA form with register coalescing. This would
eliminate the explicit SSA form destruction step, while the chordality of
(strict) SSA form interference graphs ensures coloring can be done efficiently
without explicitly building the interference graph \cite{Hack:2005:TR}.

\item Eliminate the need for explicit dependence graph construction in
instruction scheduling and software pipelining.

In the context of the AST Computing PJIT project, I have developed a post-pass
``Scoreboard'' scheduler that behaves like the dynamic instruction scheduler of
an out-of-order superscalar processor. A key point is that, given the
instructions to schedule in some priority order, such Scoreboard scheduler
computes the same instruction schedule as list scheduling, provided the
scheduled resources are used one cycle at most by the instructions. The latter
is accurately verified for most modern processors, including VLIWs such as the
ST200 family. This means that, given an instruction priority, constructing the
instruction schedule does not requires the dependence graph.

Furthermore, I have established in 2004 that modulo scheduling problems are
equivalent to the acyclic scheduling of the corresponding unwinded problem under
``regularization'' constraints \cite{Dinechin:2004:DOM}. It turns out that
Scoreboard scheduling can easily accommodate the regularization constraints, and
also be trivially extended to operate on pseudo-registers. This opens the door
to software pipeline construction within tight compilation resource constraints,
still without the expense of building an explicit dependence graph (the current
LAO software pipeliner maintains three parallel dependence graphs).

Even though a Scoreboard scheduler does not require an explicit dependence
graph, computing a good instruction priority list might require it. Given the
context of JIT compilation, one can decide to improve instruction scheduling in
by spending more time on the priority list construction. How to to best spend
time is one of the research topics we plan to explore in collaborations with
Prof. Hanen and Prof.  Munier at U. of Paris-VI.

\end{itemize}


\subsection{Predicated Execution Support}

Predicated execution support in processors comes in various flavors that can be
accommodated in compiler design in different ways:
\begin{itemize}

\item Select instructions as supported in VLIW processors such as the ST200
family are the easiest to accommodate.

\item Conditional move instructions are found in a variety of general-purpose
architectures, including DEC Alpha AMD64. Conditional move instructions
introduce ambiguous register definitions, but this can be hidden from the
compiler by using simulated Select instructions that expand into a simple copy
followed by conditional move.

\item Fully predicated instructions are found on the ARM, the IA64 and high-end
VLIW processors for media processing including TIC6xxx family, StarCore SC140
and the STMicroelectronics ST100.

Additional complexities include: there can be at most one predicate active at a
time (ARM, SC140); the need of explicitly complementing predicates (ST240);
predicate defining instructions that define multiple conditions at once (IA64,
ST100); whether the instruction set is fully predicated. For instance, the
STMicroelectronics ST240 only offers full predication on memory access
instructions; IF-converting other instructions rely on a Select instruction.

\end{itemize}

For compiler design, a key point is that after IF-conversion into Select
operations, there are no ambiguous register definitions seen by the compiler, so
textbook algorithms for SSA form and register allocation can run unchanged. On
the other hand, ambiguous (non-kill) register definitions of conditional moves
or fully predicated instructions have a major impact on code generation, in
particular: register liveness and reaching definitions; SSA form construction,
optimizations and destruction; register allocation.

Fortunately, during the design and development of the LAO for the
STMicroelectronics ST100, and later the development of the Open64 code generator
and LAO for the STMicroelectronics ST200, the difficulties of carrying
predicated code across code generator and SSA form optimizations have been
solved, in particular: the discovery of the PSI-SSA form
\cite{Stoutchinin:2001:MICRO}, that allows to carry SSA form analyzes and
optimization on predicated code; validity conditions of direct predication of
SSA form into PSI-SSA form \cite{Stoutchinin:2004:EUROPAR}; extensions of the
SSA form destruction to machine code algorithm \cite{Rastello:2004:CGO} to
handle PSI-SSA form, that have been implemented by F. de Ferri\`ere in the
Open64 code generator for the ST240.

Thus, for the JIT design discussed here, we propose to fully support predication
in a mostly processor-independent way by reusing these results. Some adaptations
will be required, in particular if we end up with SSA based register allocation.


\section{JIT Organization and Program Representations}


\subsection{CIL to Generic SSA Form}

\begin{description}

\item[Loading of CIL Representation]

\item[Verification of CIL Representation]

\item[Recovery of Pseudo-Registers] Unwind the evaluation stack and promote the
local variables.

\item[CodeRegion Formation] Includes partial inlining of methods.

\item[Generic SSA Form Construction] Based on a set of generic operators to be
defined in relation with the MDS Semantics language.

\end{description}


\subsection{Generic SSA Form to Machine SSA Form}

\begin{description}

\item[Range and Modulo Analysis] These are two basic abstract interpretations
whose purpose is to annotate program variables for the purposes of code
selection. Can be performed on the SSA form complemented with def-use chains
(as in the ST100 LAO) or on the SSI from (prototyped in the ST200 LAO).

\item[Eliminate Redundant Tests]

\item[Intrinsic Function Inlining]

\item[Loop Optimizations and Cleanups] Inner loop unrolling on the SSA form as
in the ST100 LAO and the LxBE. Induction variable identification for use in
hardware loop exploitation and auto-modifying addressing modes. Perform
SIMDization of memory accesses based on inductions and modulo analysis.

\item[Code Selection to Machine SSA Form] Should use a finite-state automaton
like GBURG? Will eventually include matching of SIMD and DSP instructions.

\end{description}


\subsection{Machine SSA Form to JIR Code}

\begin{description}

\item[Machine SSA Form Predication] The SSA form is converted to PSI-SSA form.
Use Fang's algorithm for predicate assignment.

\item[Code Cleanups and Block Alignment] Aligned basic blocks sequence provide
the basis for trace or superblock identification.

\item[Software Pipelining by Unwinding] Includes induction variable relaxation
and speculative load analysis.

\item[Register Allocation on PSI-SSA Form] On-going research at ENS-Lyon/CompSys
laboratory in collaboration with HPC compiler group in Grenoble.

\item[Scoreboard Instruction Scheduling] Prototyped in the PJIT.

\item[Conversion to JIR Code]

\end{description}


\section{Work Breakdown Structure}


\subsection{Milestone M1 (April 2006): Conversion to LIR}

\begin{description}

\item[LIR Building from PJIT]

\item[Frame Layout and ABI in LIR]

\item[Linear-Scan Register Allocation]

\item[Scoreboard Instruction Scheduling]

\item[Conversion to JIR Code]

\item[C Compiler Test Infrastructure I] Probably use PVM with Visual Studio
generated CIL and LAO within Open64 as test infrastructure.

The Open64 can be used to activate the LAO and execute its JIT optimizations in
static context. This only works for the st200 family but allows to check the
effectiveness of JIT optimizations in addition to their functionality over more
than 1 million lines of C.

\end{description}


\subsection{Milestone M2 (August 2006): Machine SSA Form}

\begin{description}

\item[Machine SSA Form Construction]

\item[Machine SSA Form Predication]

\item[Code Cleanups and Block Alignment]

\item[PSI-SSA Form Destruction]

\item[C Compiler Test Infrastructure II] Use LCC.NET code generator if
Infrastructure I is not sufficient.

\end{description}


\subsection{Milestone M3 (December 2006): Generic SSA Form}

\begin{description}

\item[Generic SSA Form Construction]

\item[Range and Modulo Analysis]

\item[Loop Optimizations and Cleanups]

\item[Code Selection to Machine SSA Form]

\item[C Compiler Test Infrastructure III] Hope to have first GCC4NET compiler by
then. Otherwise will have to extend LCC.NET to generate CIL attributes.

\end{description}


\subsection{Milestone M4 (April 2007): Static Optimizations}

\begin{description}

\item[CodeRegion Formation]

\item[Eliminate Redundant Tests]

\item[Intrinsic Function Inlining]

\item[Software Pipelining by Unwinding]

\item[Register Allocation on PSI-SSA Form]

\end{description}


\subsection{Milestone M5: Dynamic Optimizations}

\begin{description}

\item[Conditional Loop Generation]

\item[Code Specialization on Quasi-Constants]

\end{description}


\section{Requirements on the CIL Producer}

\begin{itemize}

\item Identify the CIL local variables that are in fact pseudo-registers.

\item Annotate the address variables with alignment attributes.

\item Perform extensive memory disambiguation and provide the resulting memory
dependence tables.

\item Match DSP arithmetic idioms and convert them to intrinsic calls.

\item Insert edge or path profiling in CIL and flag such code for easy removal
by the JIT.

\end{itemize}


\newcommand{\bibas}[1]{{\sc{}#1}:}
\newcommand{\bibts}[1]{\emph{#1}}
\newcommand{\bibps}[1]{{#1}}

\begin{thebibliography}{99}

\bibitem{Budimlic:2002:PLDI} \bibas{Z. Budimlic, K. Cooper, T. Harvey, K.
Kennedy, T. Oberg S. Reeves} \bibts{Fast Copy Coalescing and Live Range
Identification} \bibps{PLDI 2002, pages 25-32}. 

\bibitem{Dinechin:2000:CASES} \bibas{B. Dupont de Dinechin, F. de Ferri\`ere, C.
Guillon, A. Stoutchinin} \bibts{Code Generator Optimizations for the ST120
DSP-MCU Core} \bibps{CASES 2000}.

\bibitem{Dinechin:2004:DOM} \bibas{B. Dupont de Dinechin} \bibts{Modulo
Scheduling with Regular Unwinding} \bibps{2nd International Workshop on Discrete
Optimization Methods in Production and Logistics -- DOM-2004}, 2004.
\texttt{http://www.cri.ensmp.fr/classement/2004.html}

\bibitem{Hack:2005:TR} \bibas{S. Hack, D. Grund, G. Goos} \bibts{Towards
Register Allocation for Programs in SSA-form} \bibps{Universit�t Karlsruhe
Technical Report, September 2005}.

\bibitem{Hank:1995:MICRO} \bibas{R. E. Hank, W.-M. Hwu, B. B. R Rau}
\bibts{Region-Based Compilation: an Introduction and Motivation} \bibps{MICRO
1995: 158-168}.

\bibitem{Leung:1999:PLDI} \bibas{A. Leung, L. George} \bibts{Static Single
Assignment Form for machine Code} \bibps{ACM PLDI 1999}.

\bibitem{Paleczny:2001:JVM} \bibas{M. Paleczny, C. A. Vick, C. Click}
\bibts{The Java HotSpot Server Compiler} \bibps{Java Virtual Machine Research
and Technology Symposium 2001}.

\bibitem{Rastello:2004:CGO} \bibas{F. Rastello, F. de Ferri\`ere, C. Guillon}
\bibas{Optimizing Translation Out of SSA Using Renaming Constraints} \bibps{CGO
2004: 265-278}.

\bibitem{Sreedhar:1999:SAS} \bibas{V. C. Sreedhar, R. D.-C. Ju, D. M. Gillies,
V. Santhanam} \bibts{Translating Out of Static Single Assignment Form}
\bibps{SAS'99, LNCS 1694, pp. 194-210, 1999}.

\bibitem{Stoutchinin:2001:MICRO} \bibas{A. Stoutchinin, F. de Ferri�re}
\bibts{Efficient Static Single Assignment Form for Predication} \bibps{MICRO
2001: 172-181}.

\bibitem{Stoutchinin:2004:EUROPAR} \bibas{A. Stoutchinin, G. R. Gao}
\bibts{If-Conversion in SSA Form} \bibps{Euro-Par 2004: 336-345}.

\bibitem{Suganuma:2000:IBMSJ} \bibas{T. Suganuma, T. Ogasawara, M. Takeuchi, T.
Yasue, M. Kawahito, K. Ishizaki, H. Komatsu, T. Nakatani} \bibts{Overview of
the IBM Java Just-in-Time Compiler} \bibps{IBM Systems Journal, Volume 39,
Number 1, 2000}.

\bibitem{Suganuma:2003:PLDI} \bibas{T. Suganuma, T. Yasue, T. Nakatani} \bibts{A
Region-Based Compilation Technique for a Java Just-In-Time Compiler} \bibps{PLDI
2003: 312-323}.

\bibitem{Tabatabai:2003:INTELTJ} \bibas{A.-R. Adl-Tabatabai, J. Bharadwaj, D.-Y.
Chen, A. Ghuloum, V. Menon, B. Murphy, M. Serrano, T. Shpeisman} \bibts{The
StarJIT Compiler: A Dynamic Compiler for Managed Runtime Environments}
\bibps{Intel Technology Journal Volume 7, Number 1, February, 2003}.

\bibitem{Way:2000:PACT} \bibas{T. Way, B. Breech, L. L. Pollock} \bibts{Region
Formation Analysis with Demand-Driven Inlining for Region-Based Optimization}
\bibps{IEEE PACT 2000: 24-36}.

\bibitem{Wimmer:2005:VEE} \bibas{C. Wimmer, H. M\"oessenb\"oeck}
\bibts{Optimized Interval Splitting in a Linear Scan Register Allocator}
\bibps{pp. 132-141. VEE'05, 2005}.

\end{thebibliography}

\end{document}

